\documentclass[a4paper,11pt]{article}
\usepackage[german,italian]{babel}
\usepackage{fancyhdr}


\textwidth16cm \textheight24cm \topmargin0mm \headheight0mm
\headsep5mm \oddsidemargin0mm \evensidemargin0mm
\parindent0mm


\usepackage{amsmath}
\usepackage{parskip}
\usepackage{dsfont}
\usepackage{fullpage}
\usepackage{amssymb}
\usepackage{tikz,pgfplots}
\usepackage{cancel}
\usepackage{lmodern}

\usepackage[T1]{fontenc}

\usepackage{theorem}
\usepackage{psfrag}
\usepackage{color}
\usepackage{graphicx}
\usepackage{hyperref}

\usepackage{background}
\usepackage{keystroke}
\usepackage{etoolbox}


\makeatletter
\patchcmd{\tableofcontents}{\@starttoc{toc}}{\hypertarget{totoc}{}\@starttoc{toc}}{}{}
\makeatother

\SetBgScale{1}
\SetBgAngle{0}
\SetBgColor{black}
\SetBgPosition{current page.south}
\SetBgVshift{20pt}
\SetBgContents{\tikz[remember picture,overlay]
    \node[inner sep=0pt] {\hyperlink{totoc}{\Return}};}

\hypersetup{
    colorlinks=true,
    linkcolor=black,
    urlcolor=blue,
    pdftitle={titolo},
    pdfpagemode=FullScreen,
}




\begin{document}




\title{Matematica}

\author{Massimiliano Ferrulli}
\date{03.05.2022}



\maketitle

\section*{Coniche}
Studio di coniche, ellissi, paraboliche e iperboli

\pagebreak




\tableofcontents





\pagebreak















\section{Le Coniche }
\subsection{Sezioni coniche}
Nello spazio consideriamo due rette \textit{g} e \textit{a} incidenti nel punto V, con \( \omega \) che è l'angolo generato dall'intersezione delle due rette. 
La rotazione della retta \textit{g} attorno ad \textit{a} genera una superficie illimitata detta cono di rotazione a due falde.






\subsection{Definizione}
Una curva ottenuta come intersezione di un cono di rotazione con un piano \( \alpha \), che non passi per il vertice, è detta conica non degenere
si distinguono 3 casi a dipendenza dell'angolo di incidenta S del piano \( \alpha \) rispetto all'asse \textit{a}. 

Queste curve ottenute come sezioni di un cono, possono essere caratterizzate come luoghi geometrici, cioè come insieme di punti di un piano che soddisfano una condizione geometrica.
 




\subsubsection{Caso 1}
se il piano interseca una sola falda e taglia tutte le generatrici, cioè se \( \delta > \omega \) si ottiene una curva chiusa detta ellisse. 
In particolare se \( \alpha \perp a \) oppure \( \vec{\eta}_\alpha \backslash \backslash \, \vec{v}_a \)   \( (\delta = \frac{\pi}{2}) \) si ottiene una circonferenza
\paragraph{Luogo geometrico Ellisse}
un'ellisse è l'insieme di punti per i quali la somma delle distanze da due punti \(F_1 e F_2\) (detti fuochi) è costante.





\subsubsection{Caso 2}
se il piano \( \alpha \) interseca una sola falda ed è parallelo ad una generatrice \( ( \delta = \omega) \) si ottiene una curva aperta detta parabola
\paragraph{Luogo geometrico Parabola}
è l'insieme dei punti del piano equidistanti da un punto fisso F detto fuoco e da una retta fissa D detta direttrice






\subsubsection{Caso 3}

\(  \delta < \omega \)



\paragraph{Luogo geometrico Iperbole}
È l'insieme dei punti del piano \( \alpha \) per i quali il valore assoluto della differenza delle distanze dai due punti fissi \( F_1 e F_2\) (fuochi) è costante.
  


























\end{document}