%This is Massimiliano's latex template 
\documentclass[a4paper,11pt]{article}

\usepackage{fancyhdr}



\usepackage[english,italian]{babel}


\textwidth16cm \textheight24cm \topmargin0mm \headheight10mm
\headsep5mm \oddsidemargin0mm \evensidemargin0mm
\parindent0mm
\usepackage{bettelini}
\usepackage{subfig}
\usepackage{amsmath}
\usepackage{parskip}
\usepackage{dsfont}
\usepackage{fullpage}
\usepackage{amssymb}
\usepackage{tikz,pgfplots}
\usepackage{cancel}
\usepackage{lmodern}
\usepackage{gensymb}
\usepackage[T1]{fontenc}
\usepackage{amsfonts}
\usepackage{theorem}
\usepackage{psfrag}
\usepackage{color}
\usepackage{graphicx}
\usepackage{hyperref}
\usepackage{wrapfig}  
%\usepackage[all]{hypcap}
\usepackage{keystroke}
\usepackage{etoolbox}
\usepackage{babel}
\usepackage{siunitx}
\usepackage{float}
\usepackage{accents}









\usepackage[AMRA, 30]{masterfrontpage}

\usepackage{blindtext}

\usetikzlibrary{3d}


\pgfplotsset{
  compat=1.13,
}

\makeatletter
\patchcmd{\tableofcontents}{\@starttoc{toc}}{\hypertarget{totoc}{}\@starttoc{toc}}{}{}
\makeatother

\usepackage{fancyvrb}




\usepackage[backend=bibtex]{biblatex}
\addbibresource{references.bib}

\hypersetup{
    colorlinks=true,
    linkcolor=black,
    urlcolor=blue,
    pdftitle={Lavoro per il Concorso nazionale di Scienza e gioventù},
    pdfpagemode=FullScreen,
}
\VerbatimFootnotes
\newcommand{\overbar}[1]{\mkern 1.5mu\overline{\mkern-1.5mu#1\mkern-1.5mu}\mkern 1.5mu}





% #1    x coordinate.
\newcommand{\vandbarrier}[1]{
    \node[scale = 0.5] at (#1, 0.5) {\text{(Barriera)}};%
}

% #1    x coordinate.
\newcommand{\vabove}[1]{%
    \node[anchor = south] at (#1, 2) {\scriptsize $V = \infty$};%
}




\usetikzlibrary{decorations.pathmorphing,decorations.markings,calc} % for random steps & snake
\usetikzlibrary{arrows.meta} % for arrow size
\tikzset{>=latex} % for LaTeX arrow head
\tikzstyle{radiation}=[-{Latex[length=2,width=1.5]},red!95!black!50,opacity=0.7,very thin,decorate,
                       decoration={snake,amplitude=0.7,segment length=2,post length=2}]






\tikzset{>=latex}





          \pgfdeclareverticalshading{rainbow}{100bp}{
              color(0bp)=(red); color(25bp)=(red); color(35bp)=(yellow);
              color(45bp)=(green); color(55bp)=(cyan); color(65bp)=(blue);
              color(75bp)=(violet); color(100bp)=(violet)
            }
            
            \pgfdeclareverticalshading{blackbody}{100}{
              rgb(0)=(0,0,0);
              rgb(25)=(0,0,0);
              rgb(25+50/11*1)=(1,0.0337,0);
              rgb(25+50/11*2)=(1,0.2647,0.0033);
              rgb(25+50/11*3)=(1,0.4870,0.1411);
              rgb(25+50/11*4)=(1,0.6636,0.3583);
              rgb(25+50/11*5)=(1,0.7992,0.6045);
              rgb(25+50/11*6)=(1,0.9019,0.8473);
              rgb(25+50/11*6.5)=(1,1,1);
              rgb(25+50/11*7)=(0.9337,0.9150,1);
              rgb(25+50/11*8)=(0.7874,0.8187,1);
              rgb(25+50/11*9)=(0.6925,0.7522,1);
              rgb(25+50/11*10)=(0.6268,0.7039,1);
              rgb(75)=(0.3277,0.5022,1);
              rgb(100)=(0.3277,0.5022,1)
            }


            


\include{settings_epfl.tex}






\let\oldAA\AA
\renewcommand{\AA}{\text{\normalfont\oldAA}}


%QUICK_VECTORS  https://tex.stackexchange.com/questions/200507/inline-row-vectors-with-smallmatrix
\newcommand{\icol}[1]{% inline column vector
  \left(\begin{smallmatrix}#1\end{smallmatrix}\right)%
}

\newcommand{\irow}[1]{% inline row vector
  \left(\begin{smallmatrix}#1\end{smallmatrix}\right)%
}


\renewcommand{\iff}{
  \xleftrightarrow{\text{iff}}
}

\newcommand{\bk}[1]{
  \langle {#1} \rangle
}

\renewcommand{\mod}[1]{
  \left\lVert {#1}\right\rVert_2
}

\newcommand{\abs}[1]{
  \left\lvert {#1} \right\rvert
}

\newcommand{\rn}{
  \mathbb{R}^n
}

\newcommand{\R}{
  \mathbb{R}
}


\begin{document}


\title{Analysis II}
\subtitle{from chemistry to properties}
\author{Ferrulli Massimiliano}  
% SE VUOI LA COPERTINA: %
   % \masterfrontpage

\maketitle





\tableofcontents





\pagebreak



\section{The vector space $\mathbb{R}^n$}
\subsection{Operations}
Addition and scalar multiplications are defined as follows:
$$\mathbf{x+y} = \icol{x_1\\\vdots\\ \\ x_n} + \icol{y_1 \\ \vdots \\ \\ y_n } = \icol{x_1 + y_1 \\ \vdots \\ \\ x_n + y_n} $$
$$ \lambda \mathbf{x} = \lambda \icol{x_1 \\ \vdots \\ \\ x_n} = \icol{\lambda x_1 \\ \vdots \\ \\ \lambda x_n} $$
$$ \lambda \mathbf{x+y} = \lambda \mathbf{x} + \lambda \mathbf{y} $$
Scalar product, $ \langle \cdot , \cdot \rangle : \mathbb{R}^n \times \mathbb{R}^n \to \mathbb{R}$ , in the vector space $\mathbb{R}^n$ is defined as:
$$\langle \mathbf{x,y} \rangle = \sum_{k = 1}^{n} x_k y_k = \mathbf{x^T y} = \irow{x_1 & \cdots & x_n } \icol{y_1 \\ \vdots \\ \\ y_n}  $$
The scalar product satisfies the following properties: \\
1. Positive definiteness: $\langle \mathbf{x,x} \rangle \geq 0  \forall x $ with  $ \langle \mathbf{x,x} \rangle = 0  \iff  x = 0  $ \\
2. Simmetry: $\bk{x,y} = \bk{y,x} $ \\
3. Bilinearity: $\bk{\alpha x + \beta y, z} = \alpha\bk{x,z} + \beta\bk{y,z} \, \, \, \forall x,y,z \in \mathbb{R}^n  \, \, \text{and} \, \, \forall \alpha , \beta \in \mathbb{R} $
\subsection{Euclidean Norm on $ \mathbb{R}^n $}
The function $\mod{\cdot} : \mathbb{R}^n \to \mathbb{R} $ is defined by
$$ \mod{x} = \sqrt{\bk{x,x}} = \left(\sum_{k = 1}^{n} x^2_k\right)^{\frac{1}{2}}$$

The Euclidean norm on $\mathbb{R}^n$ has the following properties: \\
1. Non-negativity: $\|\mathrm{x}\|_2 \geqslant 0$ for all $\mathrm{x} \in \mathbb{R}^n$, with eequality $ \iff$ $\mathbf{x}=\mathbf{0}$. \\
2. Homogeneity: $\|\lambda \cdot \mathbf{x}\|_2=|\lambda| \cdot\|\mathbf{x}\|_2$ for all $\lambda \in \mathbb{R}$ and $\mathbf{x} \in \mathbb{R}^n$ \\
3. Triangle inequality: $\|\mathbf{x}+\mathbf{y}\|_2 \leqslant\|\mathbf{x}\|_2+\|\mathbf{y}\|_2 \forall \mathbf{x}, \mathbf{y} \in \mathbb{R}^n$ (also called Cauchy Schwartz inequality)\\
$$ \abs{\bk{x,y}} \leq \mod{x} \mod{y} \text{(Angle Formula)} $$
if $\mathbf{x}$ and $\mathbf{y}$ are orthogonal then:
$$ \mod{x+y}^2 = \mod{x}^2 + \mod{y}^2 \text{(Pythagoras)} $$
\underline{Definition:} \\
The Euclidean distance on $\mathbb{R}^n$ is the function $d(. \, , \, . ) : \mathbb{R}^n \times \mathbb{R}^n \to [0,\infty) $ given by:
$$ d(\mathbf{x}, \mathbf{y}):=\mod{x-y}=\sqrt{\left(x_1-y_1\right)^2+\ldots+\left(x_n-y_n\right)^2} $$
This function outputs the distance between two points in $\mathbb{R}^n$ and it satisfies the following 3 properties: \\
1. Non-negativity: $d(\mathbf{x}, \mathbf{y}) \geqslant 0$ for all $\mathbf{x}, \mathbf{y} \in \mathbb{R}^n$, with equality $\iff $ $\mathrm{x}=\mathrm{y}$. \\
2. Symmetry: $d(\mathbf{x}, \mathbf{y})=d(\mathbf{y}, \mathbf{x})$. \\
3. Triangle inequality: $d(\mathbf{x}, \mathbf{y}) \leqslant d(\mathbf{x}, \mathbf{z})+d(\mathbf{z}, \mathbf{y})$. \\
\subsection{Topology on $\mathbb{R}^n$}
\underline{Definition:} Open Ball \\
Let $\mathbf{a} \in \mathbb{R}^n$ and $r > 0$. The set 
$$B(a,r) = \{ x \in \mathbb{R}^n : d(x,a) < r \} $$
is called the open ball of radius r centered at  $ \mathbf{a} $.\\
If $\mathbb{x}$ and $\mathbb{y}$ are two distinct points then: if $\mathbf{x,y} \in \mathbb{R}^n and \mathbf{x \neq y} $ then we can find a sufficiently small open ball centered in $\mathbf{x}$ and another centered in $\mathbf{y} $ such that the two balls don't touch.
Open balls are open sets\footnote{An open set is a set with the property that if $\mathbf{x}$ is a point in the set then all points that are sufficiently near to it also belong to the set. } \\
\underline{Definition:} Open set \\
A subset $U \subseteq \mathbb{R}^n $ is open if $\forall \mathbf{x} \in U  \, \, \, \exists \varepsilon > 0 : \, \text{the open ball} \, \, B(x,\varepsilon) \, \, \text{is contained in} \, U. $  \\
\underline{Example of an open set:}\\
1. if $a < b$ are real numbers then the interval $$ (a,b) = \{x \in \mathbb{R} : a < x < b\} $$
is an open set. \\
Proof: take $r = min\{x-a, b-x\}$ (both a and b are strictly positive), the minimum is positive and the ball $B(x,r) = \{y\in\mathbb{R} : \abs{x-y} < r\}$ is a subset of $(a,b)$.
As $\mathbf{x}$ is arbitrary, that works $\forall \mathbf{x} \in (a,b)$ and so it satisfies the definition of an open set. \\
2. The infinite interval $(a,\infty)$ and $(-\infty,b)$ are also open but the intervals 
$$ (a, b]=\{x \in \mathbb{R}: a<x \leqslant b\} \quad \text { and } \quad[a, b]=\{x \in \mathbb{R}: a \leqslant x \leqslant b\} $$
are not open sets. \\
3. the rectangle $(a, b) \times(c, d)=\left\{(x, y) \in \mathbb{R}^2: a<x<b, c<y<d\right\}$ is an open set. \\
\underline{Definition:} Closed Set \\
A subset $C \subseteq \mathbb{R}^n$ is closed if its complement $\mathbb{R}^n \ C $ is open. \\
\underline{Convention:} The empty set and the space $\mathbb{R}^n $ are the only two spaces both open and closed at the same time. \\
\underline{Definition:} Closed Ball \\
Let $\mathbf{a } \in \mathbb{R}^n $ and $r > 0$. The set 
$$\overline{B(\mathbf{a},r)} = \{ \mathbf{x} \in \mathbb{R}^n : d(\mathbf{x,a}) \leq r\} $$
is called the closed ball of radius r centered at $\mathbf{a} $ and it is a closed set.
\underline{Example of a closed set:} \\
1. The closed interval $[a,b] = \{x \in \mathbb{R} \, : \, a \leq x \leq b\}$ is a closed set and its complementary $\mathbb{R} \ [a,b] = (-\infty,a) \cup (b,\infty) $ is an open set. \\
2. Infinite intervals with closed boundary $[a,\infty)$ and $(-\infty,b] $ are closed sets. \\
3. Halfopen intervals such as $[a,b)$ or $(a,b]$ \underline{are neither closed nor open sets.} \\
4. Any set consisting of only finitely many points is a closed set. \\
\underline{Propositions:} \\
$\cdot$ if $U \subseteq \mathbb{R}^n$ is open and $C \subseteq \mathbb{R}^n$ is closed then $U \ C$ is open.  (Open \ Closed = Open)   \\
$\cdot$ if $U \subseteq \mathbb{R}^n$ is closed and $C \subseteq \mathbb{R}^n$ is open then $U \ C$ is closed.  (Closed \ Open = Closed )  \\
$\cdot$ if $U_1 , \cdots \, U_k \subseteq \mathbb{R}^n $ are open then $U_1 \cup \cdots \cup U_k$ and $U_1 \cap \cdots \cap U_k$ are open. \\
$\cdot$ if $C_1 , \cdots \, C_k \subseteq \mathbb{R}^n $ are open then $C_1 \cup \cdots \cup C_k$ and $C_1 \cap \cdots \cap C_k$ are closed\footnote{intersezione e unione di sottoinsiemi chiusi (rispettivamente aperti) danno un sottoinsieme ancora chiuso (rispettivamente aperto).}. \\ 
\underline{Definition:} Let S be a subset of $\mathbb{R}^n$ and $\mathbf{x}$ a point in $\mathbb{R}^n$ \\
$\cdot$ We call $\mathbf{x}$ an interior point of S if $ \exists r > 0 : $ the ball $B(x,r)$ is contained in S. \\
$\cdot$ We call $\mathbf{x}$ an exterior point of S if  $ \exists r > 0 : $ the ball $B(x,r)$ has empty intersection with S \\
$\cdot$ We call $\mathbf{x}$ an boundary point of S if  $ \exists r > 0 : $ the ball $B(x,r)$ if it is neither an interior point neither an exterior point. $\mathbf{x}$ is a boundary point if $\forall r > 0$ the ball $B(x,r) $ has non empty intersection with S without being entirely contained in S. \\  \\

 \begin{figure}[!h]
   \centering
   \subfloat[Schematic for the 3 types of points]{
     \includegraphics[width=50mm]{images/Int_ext_bound_points.png}
   }
 \end{figure}

\underline{Definition:} Interior \\
The set of all interior points of a set S is called the interior of S and it is denoted by $\mathbf{\mathring{S}}$ and it's the largest open set contained inside of S  \\
\underline{Definition:} Boundary \\
The set of all boundary points of a set S is called the boundary of S and we use $\mathbf{\partial S} $ to denote it \\
\underline{Definition:} Closure \\
The closure of S is denoted by $\overline{S}$ and it's the set of points $\mathbf{x} \in \mathbb{R}^n$ with the property that $\forall r > 0 \, \, , \, \, B(x,r) \cap S \neq \emptyset  $ . The closure of S is the union of all its interior points and boundary points. $\mathring{S}$  is the smallest closed set that contains S. \\
$$ \mathring{S} \subseteq \overline{S} \subseteq S $$ 
$$ \mathring{S} = S \cap \partial S \hspace{5mm} \overline{S} = S \cup \partial S \hspace{5mm} \partial S = \overline{S} \cap \mathring{S} $$ 

\begin{figure}[!h]
  \centering
  \subfloat[Schematic for the 3 types of points]{
    \includegraphics[width=50mm]{images/types_sets.png}
  }
\end{figure}

\underline{Corollary:} \\
$\cdot$ A set S is open $\iff$ $ S = \mathring{S} $ \\
$\cdot$ A set S is closed $\iff$ $S = \overline{S} $ \\

\underline{Examples:} \\ 

\subsection{Convergence of sequences in $\mathbb{R}^n$}
\underline{Definition:} sequences in $\mathbb{R}^n$ \\
A sequence of elements of $\mathbb{R}^n$ is a function $ k \mapsto \mathbb{x}_k $ that associates to every $k \in \mathbb{N} $ an element $\mathbf{x}_i \in \mathbb{R}^n$ \\
\underline{Convention:} We denote $\left(\mathbb{x}_k\right)_{k \in \mathbb{N}} $ a sequence in $\mathbb{R}^n$ and we can consider each coordinate as an individual sequence.
$$ \left(\mathbf{x}_k\right)_{k \in \mathbb{N}} = \icol{ \left( x_{1,k}\right)_{k \in \mathbb{N}} \\ \vdots \\ \\ \left(x_{n,k}\right)_{k \in \mathbb{N}} } $$
\underline{Definition:} Convergent sequence \\
A sequence $\left(\mathbf{x}_k\right)_{k \in \mathbb{N}}$ of points in $\mathbb{R}^n$ converges to a point $\mathbf{x} \in \mathbb{R}^n$ if $\forall \varepsilon > 0 \exists N > 1 : $ when $ k \geq N $, then $d(\mathbf{x}_k, \mathbf{x}) < \varepsilon $. \\
\underline{Convention:} We call $\mathbf{x} $ the limit of $\left(\mathbf{x}_k\right)_{k \in \mathbb{N}}$ and write:
$$ \lim_{k \to \infty} \mathbf{x}_k = \mathbf{x} $$
$\cdot$ If the limit exists, then it's unique. \\
$\cdot$ A sequence $\left(\mathbf{x}_k\right)_{k \in \mathbb{N}}$ converges to $\mathbf{x}$ $\iff$ the distance $d(\mathbf{x}_k,\mathbf{x})$ converges to 0.
$$ \lim_{k \to \infty} \mathbf{x}_k = \mathbf{x} \longleftrightarrow \lim_{k \to \infty} d(\mathbf{x}_k,\mathbf{x}) = 0 $$
$\cdot$ A sequence converges to $\mathbf{x}$ $\iff$ each coordinate of $\left(\mathbf{x}_k\right)_{k \in \mathbb{N}}$ \underline{converges to the respective coordinate} of $\mathbf{x}$ 

$$
\begin{aligned}
&\left(\mathbf{x}_k\right)_{k \in \mathbb{N}}=\left(\begin{array}{c}
\left(x_{1, k}\right)_{k \in \mathbb{N}} \\
\vdots \\
\left(x_{n, k}\right)_{k \in \mathbb{N}}
\end{array}\right) \quad \text { and } \quad \mathbf{x}=\left(\begin{array}{c}
x_1 \\
\vdots \\
x_n
\end{array}\right)\\
&\lim _{k \rightarrow+\infty} \mathbf{x}_k=\mathbf{x} \Longleftrightarrow \lim _{k \rightarrow+\infty} x_{i, k}=x_i \hspace{3mm} \forall i=1, \ldots, n
\end{aligned}
$$

\underline{Theorem:} Vector space arithmetic of limits of sequences\\
Let $\left(\mathbf{x}_k\right)_{k \in \mathbb{N}}$ and $\left(\mathbf{y}_k\right)_{k \in \mathbb{N}}$ be sequences in $\mathbb{R}^n$ and let $\left(\lambda_k\right)_{k \in \mathbb{R}}$ be a sequence in $\mathbb{R}$.
(i) If $\left(\mathbf{x}_k\right)_{k \in \mathbb{N}}$ and $\left(\mathbf{y}_k\right)_{k \in \mathbb{N}}$ both converge then so does $\left(\mathbf{x}_k+\mathbf{y}_k\right)_{k \in \mathbb{N}}$ and
$$
\lim _{k \rightarrow+\infty} \mathbf{x}_k+\mathbf{y}_k=\lim _{k \rightarrow+\infty} \mathbf{x}_k+\lim _{k \rightarrow+\infty} \mathbf{y}_k .
$$
(ii) If $\left(\mathbf{x}_k\right)_{k \in \mathbb{N}}$ and $\left(\lambda_k\right)_{k \in \mathbb{N}}$ both converge then so does $\left(\lambda_k \mathbf{x}_k\right)_{k \in \mathbb{N}}$ and
$$
\lim _{k \rightarrow+\infty} \lambda_k \mathbf{x}_k=\left(\lim _{k \rightarrow+\infty} \lambda_k\right) \cdot\left(\lim _{k \rightarrow+\infty} \mathbf{y}_k\right) .
$$
(iii) If $\left(\mathbf{x}_k\right)_{k \in \mathbb{N}}$ and $\left(\mathbf{y}_k\right)_{k \in \mathbb{N}}$ both converge then so does $\left(\left\langle\mathbf{x}_k, \mathbf{y}_k\right\rangle\right)_{k \in \mathbb{N}}$ and
$$
\lim _{k \rightarrow+\infty}\left\langle\mathbf{x}_k, \mathbf{y}_k\right\rangle=\left\langle\lim _{k \rightarrow+\infty} \mathbf{x}_k, \lim _{k \rightarrow+\infty} \mathbf{y}_k\right\rangle
$$
\underline{Definition:} Cauchy sequences\\
A sequence $\left(\mathbf{x}_k\right)_{k \in \mathbb{N}}$ is a Cauchy sequence if $ \forall \varepsilon > 0 \exists N > 1 : \, \, k,l \geq N $ implies $d(\mathbf{x}_k, \mathbf{x}_l) < \varepsilon$ and so it means that evey Cauchy sequence is a convergent sequence and viceversa. \\
\underline{Proposition:} Let $S \subseteq \mathbb{R}^n $ be a non empty set and suppose $\mathbf{x} \in \partial S $ (Boundary set), then $\exists$ a sequence of elements $ \in S \, , \, \mathbf{x}_1, \mathbf{x}_2 \cdots \in S \, , \, : $
$$ \lim_{k \to \infty} \mathbf{x}_k = \mathbf{x} $$
\underline{Proposition:}  \\
Let $ C \subseteq \mathbb{R}^n $ be a closed set and let $\left(\mathbf{x}_k\right)_{k \in \mathbb{N}}$ be a sequence of elements in C. if $\left(\mathbf{x}_k\right)_{k \in \mathbb{N}}$ converges then the limit $ \lim_{k \to \infty } \mathbf{x}_k = \mathbf{x} $ must belong to C. \\
\underline{Definition:} Bounded set \\
A subset $ E \subseteq \mathbb{R}^n $ is bounded if it is contained in a ball of finite radius centered at the origin: 
$$ E \subseteq B(0,r) \text{for some} R < \infty $$
A closed set don't need to be bounded, $ [0,\infty) $ by convention is closed but not bounded. \\
\underline{Definition:} Compact set: \\
A subset $ C \subseteq \rn $ is compact if it is closed and bounded. \\
\underline{Definition:} Subsequence \\
A subsequence $ \left(\mathbf{x}_k\right)_{k \in \R} $ is any sequence of the form $ \left(\mathbf{x}_{k_i}\right)_{i \in \R} $ where $\left(k_i\right)_{i \in \R} $ is a strictly increasing sequence of positive integers. \\
If a sequence converges then any subsequence of it converges to the same limit. \\
\underline{Theorem:} Bolzano-Weierstrass theorem in $\rn$ \\
Let $C \subseteq \rn $ be compact. Any sequence $ \left(\mathbf{x}_k\right)_{k \in \R} $ of elements in C posseses a convergent subsequence $ \left(\mathbf{x}_{k_i}\right)_{i \in \R}$ whose limit is  in C. \\
\underline{Definition:} Bounded sequence in $\rn$ \\
A sequence $ \left(\mathbf{x}_k\right)_{k \in \R} $  is bounded if $\exists C > 0 : \mod{\mathbf{x}_k} \leq C \, \, \, \forall k \in \mathbb{N} $ (C a constant). \\
Every convergent sequence is bounded but non viceversa, for example $ x_k = \left(-1\right)^k $ is bounded but not convergent. \\ 
\underline{Corollary} (Consequence of the previous definition and bolzano-Weierstrass theorem) \\
Each bounded sequence $ \left(\mathbf{x}_k\right)_{k \in \R}  $ in $\rn$ has a convergent susbequence $ \left(\mathbf{x}_{k_i}\right)_{i \in \R} $ \\
\subsection{Paths and Path-Connected Sets}
\underline{Definition:} Path \\
Let $ I \subseteq \rn $ be an interval. A \underline{path} (or \underline{curve}) in $\rn$ is a function $ f \, \, : \, \, I \to \rn $ with:
$$ f(t) = \icol{f_1(t) \\ \vdots \\ \\ f_n(t)} $$
where  $ f_i \, \, : \, \, I \to \R $ is continuos $ \forall i = 1 , \, \, \cdots  \, \, , n $. A path is a vector with every component as a continuos function from an interval $I$ to $\R$ \\
\underline{Definition:} Path connected sets\\
Let $E \subseteq \rn $. We say E is \underline{path-connected} if $\forall \mathbf{x,y} \in E \, \, \exists $ a path  $f : [0,1] \to E $ with $f(0) = \mathbf{x} $ and $f(1) = \mathbf{y} $. \\
$f : [0,1] \to E $ means that $ Im(f) \subseteq E $.






\pagebreak



\section{Allegati}
\label{sec:allegati}
\subsection{Dimostrazione 1}
\label{sec:Dimostrazione 1}













\end{document} 