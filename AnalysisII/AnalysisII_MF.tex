%This is Massimiliano's latex template 
\documentclass[a4paper,11pt]{article}

\usepackage{fancyhdr}



\usepackage[english,italian]{babel}


\textwidth16cm \textheight24cm \topmargin0mm \headheight10mm
\headsep5mm \oddsidemargin0mm \evensidemargin0mm
\parindent0mm
\usepackage{bettelini}
\usepackage{subfig}
\usepackage{amsmath}
\usepackage{parskip}
\usepackage{dsfont}
\usepackage{fullpage}
\usepackage{amssymb}
\usepackage{tikz,pgfplots}
\usepackage{cancel}
\usepackage{lmodern}
\usepackage{gensymb}
\usepackage[T1]{fontenc}
\usepackage{amsfonts}
\usepackage{theorem}
\usepackage{psfrag}
\usepackage{color}
\usepackage{graphicx}
\usepackage{hyperref}
\usepackage{wrapfig}  
%\usepackage[all]{hypcap}
\usepackage{keystroke}
\usepackage{etoolbox}
\usepackage{babel}
\usepackage{siunitx}
\usepackage{float}










\usepackage[AMRA, 30]{masterfrontpage}

\usepackage{blindtext}

\usetikzlibrary{3d}


\pgfplotsset{
  compat=1.13,
}

\makeatletter
\patchcmd{\tableofcontents}{\@starttoc{toc}}{\hypertarget{totoc}{}\@starttoc{toc}}{}{}
\makeatother

\usepackage{fancyvrb}




\usepackage[backend=bibtex]{biblatex}
\addbibresource{references.bib}

\hypersetup{
    colorlinks=true,
    linkcolor=black,
    urlcolor=blue,
    pdftitle={Lavoro per il Concorso nazionale di Scienza e gioventù},
    pdfpagemode=FullScreen,
}
\VerbatimFootnotes
\newcommand{\overbar}[1]{\mkern 1.5mu\overline{\mkern-1.5mu#1\mkern-1.5mu}\mkern 1.5mu}





% #1    x coordinate.
\newcommand{\vandbarrier}[1]{
    \node[scale = 0.5] at (#1, 0.5) {\text{(Barriera)}};%
}

% #1    x coordinate.
\newcommand{\vabove}[1]{%
    \node[anchor = south] at (#1, 2) {\scriptsize $V = \infty$};%
}




\usetikzlibrary{decorations.pathmorphing,decorations.markings,calc} % for random steps & snake
\usetikzlibrary{arrows.meta} % for arrow size
\tikzset{>=latex} % for LaTeX arrow head
\tikzstyle{radiation}=[-{Latex[length=2,width=1.5]},red!95!black!50,opacity=0.7,very thin,decorate,
                       decoration={snake,amplitude=0.7,segment length=2,post length=2}]






\tikzset{>=latex}





          \pgfdeclareverticalshading{rainbow}{100bp}{
              color(0bp)=(red); color(25bp)=(red); color(35bp)=(yellow);
              color(45bp)=(green); color(55bp)=(cyan); color(65bp)=(blue);
              color(75bp)=(violet); color(100bp)=(violet)
            }
            
            \pgfdeclareverticalshading{blackbody}{100}{
              rgb(0)=(0,0,0);
              rgb(25)=(0,0,0);
              rgb(25+50/11*1)=(1,0.0337,0);
              rgb(25+50/11*2)=(1,0.2647,0.0033);
              rgb(25+50/11*3)=(1,0.4870,0.1411);
              rgb(25+50/11*4)=(1,0.6636,0.3583);
              rgb(25+50/11*5)=(1,0.7992,0.6045);
              rgb(25+50/11*6)=(1,0.9019,0.8473);
              rgb(25+50/11*6.5)=(1,1,1);
              rgb(25+50/11*7)=(0.9337,0.9150,1);
              rgb(25+50/11*8)=(0.7874,0.8187,1);
              rgb(25+50/11*9)=(0.6925,0.7522,1);
              rgb(25+50/11*10)=(0.6268,0.7039,1);
              rgb(75)=(0.3277,0.5022,1);
              rgb(100)=(0.3277,0.5022,1)
            }


            


%%%%%%%%%%%%%%%%%%%%%%%%%%%%%%%%%%%%%%%%%%%%%%
%
%		Thesis Settings
%
%		EDOC Template
%		2011
%
%%%%%%%%%%%%%%%%%%%%%%%%%%%%%%%%%%%%%%%%%%%%%%


\usepackage[T1]{fontenc}
\usepackage[utf8]{inputenc}


% % Uncomment for bibliography
% % Bibliography using Biblatex
%\usepackage{doi}
%\usepackage[autostyle]{csquotes}
% \usepackage[
%     backend=biber,
%     style=authoryear,
%     natbib=true,
%     firstinits=true,
%     sortlocale=en_US,
%     url=false, 
%     doi=true,
%     eprint=false,
%     isbn=false
% ]{biblatex}
%\addbibresource{tail/bibliography.bib}
% % OR Bibliography management for Bibtex 
% Load natbib before babel
%\usepackage[round]{natbib}





%%%%%%%%%%%%%%%%%%%%%%%%%%%%%%%%%%%%%%%%%%%%%%%
%% EDOC THESIS TEMPLATE: Variant 1.0 -> Latin modern, large text width&height
%%%%%%%%%%%%%%%%%%%%%%%%%%%%%%%%%%%%%%%%%%%%%%%
\usepackage{lmodern} % use this to fix blurry typewriter text font
%\usepackage[a4paper,top=22mm,bottom=28mm,inner=35mm,outer=25mm]{geometry}
%%%%%%%%%%%%%%%%%%%%%%%%%%%%%%%%%%%%%%%%%%%%%%%

%%%%%%%%%%%%%%%%%%%%%%%%%%%%%%%%%%%%%%%%%%%%%%
% EDOC THESIS TEMPLATE: Variant 2.0 -> Utopia, Gabarrit A (lighter pages)
%%%%%%%%%%%%%%%%%%%%%%%%%%%%%%%%%%%%%%%%%%%%%%
\usepackage{fourier} % Utopia font-typesetting including mathematical formula compatible with newer TeX-Distributions (>2010)
%\usepackage{utopia} % on older systems -> use this package instead of fourier in combination with mathdesign for better looking results
%\usepackage[adobe-utopia]{mathdesign}
\setlength{\textwidth}{146.8mm} % = 210mm - 37mm - 26.2mm
\setlength{\oddsidemargin}{11.6mm} % 37mm - 1in (from hoffset) piu grande e meno margine
\setlength{\evensidemargin}{0.8mm} % = 26.2mm - 1in (from hoffset)
\setlength{\topmargin}{-5.2mm} % = 0mm -1in + 23.2mm 
\setlength{\textheight}{221.9mm} % = 297mm -29.5mm -31.6mm - 14mm (12 to accomodate footline with pagenumber)
\setlength{\headheight}{14pt}
%%%%%%%%%%%%%%%%%%%%%%%%%%%%%%%%%%%%%%%%%%%%%%


\usepackage{setspace} % increase interline spacing slightly
\setstretch{1.1}

\makeatletter
\setlength{\@fptop}{0pt}  % for aligning all floating figures/tables etc... to the top margin
\makeatother


\usepackage{graphicx  }
\graphicspath{{images/}}

\usepackage{subfig}
\usepackage{booktabs}
\usepackage{lipsum}
\usepackage{microtype}
\usepackage{url}
\usepackage{mathtools,halloweenmath}
\usetikzlibrary{patterns}
\usetikzlibrary{arrows.meta, calc, decorations.markings, quotes,positioning}
\usepackage{fancyhdr}
\usepackage{geometry}
\renewcommand{\sectionmark}[1]{\markright{\thesection\ #1}}
\pagestyle{fancy}
	\fancyhf{}
	\renewcommand{\headrulewidth}{0.4pt}
	\renewcommand{\footrulewidth}{0pt}
  \setlength{\headsep}{10pt}
  \fancyhead[OC]{\bfseries \rightmark}
	\fancyhead[EL]{\bfseries \nouppercase{\leftmark}}
  \fancyhead[OR]{\thepage}
  \fancyhead[OL]{Ferrulli Massimiliano}
\fancypagestyle{plain}{
	\fancyhf{}
	\renewcommand{\headrulewidth}{0pt}
	\renewcommand{\footrulewidth}{0pt}
	\fancyfoot[EL,OR]{\thepage}}
\fancypagestyle{addpagenumbersforpdfimports}{
	\fancyhead{}
	\renewcommand{\headrulewidth}{0pt}
	\fancyfoot{}
	\fancyfoot[RO,LE]{\thepage}
}

%\geometry{
%    top=10cm
%}

\usepackage{listings}
\lstset{language=[LaTeX]Tex,tabsize=4, basicstyle=\scriptsize\ttfamily, showstringspaces=false, numbers=left, numberstyle=\tiny, numbersep=10pt, breaklines=true, breakautoindent=true, breakindent=10pt}

\usepackage{hyperref}
\hypersetup{pdfborder={0 0 0},
	colorlinks=true,
	linkcolor=black,
	citecolor=black,
	urlcolor=black}
\urlstyle{same}
\ifpdf
\usepackage[final]{pdfpages}
\else
\usepackage{calc}
\usepackage{breakurl}
\usepackage[nlwarning=false]{hypdvips}
\usepackage{backref}
\renewcommand*{\backref}[1]{}
\fi
\usepackage{bookmark}

\makeatletter
\renewcommand\@pnumwidth{20pt}
\makeatother

%%%%%%%%%%%%%%% OCIOOOO A QUESTO %%%%%%%%%%%%%%%%%%%%%%%%
\makeatletter
\def\cleardoublepage{\clearpage\if@twoside \ifodd\c@page\else 
    \hbox{}
    \thispagestyle{empty}
    \if@twocolumn\hbox{}\newpage\fi\fi\fi}
\makeatother \clearpage{\pagestyle{plain}\cleardoublepage}


%%%%% CHAPTER HEADER %%%%
\usepackage{color}
\usepackage{tikz}
\usepackage[explicit]{titlesec}
\newcommand*\chapterlabel{}
%\renewcommand{\thechapter}{\Roman{chapter}}
\titleformat{\chapter}[display]  % type (section,chapter,etc...) to vary,  shape (eg display-type)
	{\normalfont\bfseries\Huge} % format of the chapter
	{\gdef\chapterlabel{\thechapter\ }}     % the label 
 	{0pt} % separation between label and chapter-title
 	  {\begin{tikzpicture}[remember picture,overlay]
    \node[yshift=-8cm] at (current page.north west)
      {\begin{tikzpicture}[remember picture, overlay]
        \draw[fill=black] (0,0) rectangle(35.5mm,15mm);
        \node[anchor=north east,yshift=-7.2cm,xshift=34mm,minimum height=30mm,inner sep=0mm] at (current page.north west)
        {\parbox[top][30mm][t]{15mm}{\raggedleft \rule{0cm}{0.6cm}\color{white}\chapterlabel}};  %the empty rule is just to get better base-line alignment
        \node[anchor=north west,yshift=-7.2cm,xshift=37mm,text width=\textwidth,minimum height=30mm,inner sep=0mm] at (current page.north west)
              {\parbox[top][30mm][t]{\textwidth}{\rule{0cm}{0.6cm}\color{black}#1}};
       \end{tikzpicture}
      };
   \end{tikzpicture}
   \gdef\chapterlabel{}
  } % code before the title body

\newcounter{myparts}
\newcommand*\partlabel{}
\titleformat{\part}[display]  % type (section,chapter,etc...) to vary,  shape (eg display-type)
	{\normalfont\bfseries\Huge} % format of the part
	{\gdef\partlabel{\thepart\ }}     % the label 
 	{0pt} % separation between label and part-title
 	  {\ifpdf\setlength{\unitlength}{20mm}\else\setlength{\unitlength}{0mm}\fi
	  \addtocounter{myparts}{1}
	  \begin{tikzpicture}[remember picture,overlay]
    \node[anchor=north west,xshift=-65mm,yshift=-6.9cm-\value{myparts}*20mm] at (current page.north east) % for unknown reasons: 3mm missing -> 65 instead of 62
      {\begin{tikzpicture}[remember picture, overlay]
        \draw[fill=black] (0,0) rectangle(62mm,20mm);   % -\value{myparts}\unitlength
        \node[anchor=north west,yshift=-6.1cm-\value{myparts}*\unitlength,xshift=-60.5mm,minimum height=30mm,inner sep=0mm] at (current page.north east)
        {\parbox[top][30mm][t]{55mm}{\raggedright \color{white}Part \partlabel \rule{0cm}{0.6cm}}};  %the empty rule is just to get better base-line alignment
        \node[anchor=north east,yshift=-6.1cm-\value{myparts}*\unitlength,xshift=-63.5mm,text width=\textwidth,minimum height=30mm,inner sep=0mm] at (current page.north east)
              {\parbox[top][30mm][t]{\textwidth}{\raggedleft \rule{0cm}{0.6cm}\color{black}#1}};
       \end{tikzpicture}
      };
   \end{tikzpicture}
   \gdef\partlabel{}
  } % code before the title body


\usepackage[AMRA, 30]{masterfrontpage}
\usepackage{amsmath}
\usepackage{amsfonts}
\usepackage{amssymb}
\usepackage{mathtools}


\graphicspath{ {figures/} }
\usepackage{array}



% Fix the problem with delimiter size caused by fourier and amsmath packges.
\makeatletter
\def\resetMathstrut@{%
  \setbox\z@\hbox{%
    \mathchardef\@tempa\mathcode`\(\relax
      \def\@tempb##1"##2##3{\the\textfont"##3\char"}%
      \expandafter\@tempb\meaning\@tempa \relax
  }%
  \ht\Mathstrutbox@1.2\ht\z@ \dp\Mathstrutbox@1.2\dp\z@
}
\makeatother






\let\oldAA\AA
\renewcommand{\AA}{\text{\normalfont\oldAA}}


%QUICK_VECTORS  https://tex.stackexchange.com/questions/200507/inline-row-vectors-with-smallmatrix
\newcommand{\icol}[1]{% inline column vector
  \left(\begin{smallmatrix}#1\end{smallmatrix}\right)%
}

\newcommand{\irow}[1]{% inline row vector
  \left(\begin{smallmatrix}#1\end{smallmatrix}\right)%
}


\renewcommand{\iff}{
  \xleftrightarrow{\text{iff}}
}

\newcommand{\bk}[1]{
  \langle {#1} \rangle
}

\renewcommand{\mod}[1]{
  \left\lVert {#1}\right\rVert_2
}

\newcommand{\abs}[1]{
  \left\lvert {#1} \right\rvert
}





\begin{document}


\title{Analysis II}
\subtitle{from chemistry to properties}
\author{Ferrulli Massimiliano}  
% SE VUOI LA COPERTINA: %
   % \masterfrontpage

\maketitle





\tableofcontents





\pagebreak



\section{The vector space $\mathbb{R}^n$}
\subsection{Operations}
Addition and scalar multiplications are defined as follows:
$$\mathbf{x+y} = \icol{x_1\\\vdots\\ \\ x_n} + \icol{y_1 \\ \vdots \\ \\ y_n } = \icol{x_1 + y_1 \\ \vdots \\ \\ x_n + y_n} $$
$$ \lambda \mathbf{x} = \lambda \icol{x_1 \\ \vdots \\ \\ x_n} = \icol{\lambda x_1 \\ \vdots \\ \\ \lambda x_n} $$
$$ \lambda \mathbf{x+y} = \lambda \mathbf{x} + \lambda \mathbf{y} $$
Scalar product, $ \langle \cdot , \cdot \rangle : \mathbb{R}^n \times \mathbb{R}^n \to \mathbb{R}$ , in the vector space $\mathbb{R}^n$ is defined as:
$$\langle \mathbf{x,y} \rangle = \sum_{k = 1}^{n} x_k y_k = \mathbf{x^T y} = \irow{x_1 & \cdots & x_n } \icol{y_1 \\ \vdots \\ \\ y_n}  $$
The scalar product satisfies the following properties: \\
1. Positive definiteness: $\langle \mathbf{x,x} \rangle \geq 0  \forall x $ with  $ \langle \mathbf{x,x} \rangle = 0  \iff  x = 0  $ \\
2. Simmetry: $\bk{x,y} = \bk{y,x} $ \\
3. Bilinearity: $\bk{\alpha x + \beta y, z} = \alpha\bk{x,z} + \beta\bk{y,z} \, \, \, \forall x,y,z \in \mathbb{R}^n  \, \, \text{and} \, \, \forall \alpha , \beta \in \mathbb{R} $
\subsection{Euclidean Norm on $ \mathbb{R}^n $}
The function $\mod{\cdot} : \mathbb{R}^n \to \mathbb{R} $ is defined by
$$ \mod{x} = \sqrt{\bk{x,x}} = \left(\sum_{k = 1}^{n} x^2_k\right)^{\frac{1}{2}}$$

The Euclidean norm on $\mathbb{R}^n$ has the following properties: \\
1. Non-negativity: $\|\mathrm{x}\|_2 \geqslant 0$ for all $\mathrm{x} \in \mathbb{R}^n$, with eequality $ \iff$ $\mathbf{x}=\mathbf{0}$. \\
2. Homogeneity: $\|\lambda \cdot \mathbf{x}\|_2=|\lambda| \cdot\|\mathbf{x}\|_2$ for all $\lambda \in \mathbb{R}$ and $\mathbf{x} \in \mathbb{R}^n$ \\
3. Triangle inequality: $\|\mathbf{x}+\mathbf{y}\|_2 \leqslant\|\mathbf{x}\|_2+\|\mathbf{y}\|_2 \forall \mathbf{x}, \mathbf{y} \in \mathbb{R}^n$ (also called Cauchy Schwartz inequality)\\
$$ \abs{\bk{x,y}} \leq \mod{x} \mod{y} \text{(Angle Formula)} $$
if $\mathbf{x}$ and $\mathbf{y}$ are orthogonal then:
$$ \mod{x+y}^2 = \mod{x}^2 + \mod{y}^2 \text{(Pythagoras)} $$
\underline{Definition:} \\
The Euclidean distance on $\mathbb{R}^n$ is the function $d(. \, , \, . ) : \mathbb{R}^n \times \mathbb{R}^n \to [0,\infty) $ given by:
$$ d(\mathbf{x}, \mathbf{y}):=\mod{x-y}=\sqrt{\left(x_1-y_1\right)^2+\ldots+\left(x_n-y_n\right)^2} $$
This function outputs the distance between two points in $\mathbb{R}^n$ and it satisfies the following 3 properties: \\
1. Non-negativity: $d(\mathbf{x}, \mathbf{y}) \geqslant 0$ for all $\mathbf{x}, \mathbf{y} \in \mathbb{R}^n$, with equality $\iff $ $\mathrm{x}=\mathrm{y}$. \\
2. Symmetry: $d(\mathbf{x}, \mathbf{y})=d(\mathbf{y}, \mathbf{x})$. \\
3. Triangle inequality: $d(\mathbf{x}, \mathbf{y}) \leqslant d(\mathbf{x}, \mathbf{z})+d(\mathbf{z}, \mathbf{y})$. \\
\subsection{Topology on $\mathbb{R}^n$}
\underline{Definition:} Open Ball \\
Let $\mathbf{a} \in \mathbb{R}^n$ and $r > 0$. The set 
$$B(a,r) = \{ x \in \mathbb{R}^n : d(x,a) < r \} $$
is called the open ball of radius r centered at  $ \mathbf{a} $.\\
If $\mathbb{x}$ and $\mathbb{y}$ are two distinct points then: if $\mathbf{x,y} \in \mathbb{R}^n and \mathbf{x \neq y} $ then we can find a sufficiently small open ball centered in $\mathbf{x}$ and another centered in $\mathbf{y} $ such that the two balls don't touch.
Open balls are open sets\footnote{An open set is a set with the property that if $\mathbf{x}$ is a point in the set then all points that are sufficiently near to it also belong to the set. } \\
\underline{Definition:} Open set \\
A subset $U \subseteq \mathbb{R}^n $ is open if $\forall \mathbf{x} \in U  \, \, \, \exists \varepsilon > 0 : \, \text{the open ball} \, \, B(x,\varepsilon) \, \, \text{is contained in} \, U. $  \\
\underline{Example of an open set:}\\
1. if $a < b$ are real numbers then the interval $$ (a,b) = \{x \in \mathbb{R} : a < x < b\} $$
is an open set. \\
Proof: take $r = min\{x-a, b-x\}$ (both a and b are strictly positive), the minimum is positive and the ball $B(x,r) = \{y\in\mathbb{R} : \abs{x-y} < r\}$ is a subset of $(a,b)$.
As $\mathbf{x}$ is arbitrary, that works $\forall \mathbf{x} \in (a,b)$ and so it satisfies the definition of an open set. \\
2. The infinite interval $(a,\infty)$ and $(-\infty,b)$ are also open but the intervals 
$$ (a, b]=\{x \in \mathbb{R}: a<x \leqslant b\} \quad \text { and } \quad[a, b]=\{x \in \mathbb{R}: a \leqslant x \leqslant b\} $$
are not open sets. \\
3. the rectangle $(a, b) \times(c, d)=\left\{(x, y) \in \mathbb{R}^2: a<x<b, c<y<d\right\}$ is an open set. \\
\underline{Definition:} Closed Set \\
A subset $C \subseteq \mathbb{R}^n$ is closed if its complement $\mathbb{R}^n \ C $ is open. \\
\underline{Convention:} The empty set and the space $\mathbb{R}^n $ are the only two spaces both open and closed at the same time. \\
\underline{Definition:} Closed Ball \\
Let $\mathbf{a } \in \mathbb{R}^n $ and $r > 0$. The set 
$$\overline{B(\mathbf{a},r)} = \{ \mathbf{x} \in \mathbb{R}^n : d(\mathbf{x,a}) \leq r\} $$
is called the closed ball of radius r centered at $\mathbf{a} $ and it is a closed set.
\underline{Example of a closed set:} \\
1. The closed interval $[a,b] = \{x \in \mathbb{R} \, : \, a \leq x \leq b\}$ is a closed set and its complementary $\mathbb{R} \ [a,b] = (-\infty,a) \cup (b,\infty) $ is an open set. \\
2. Infinite intervals with closed boundary $[a,\infty)$ and $(-\infty,b] $ are closed sets. \\
3. Halfopen intervals such as $[a,b)$ or $(a,b]$ \underline{are neither closed nor open sets.} \\
4. Any set consisting of only finitely many points is a closed set. \\
\underline{Propositions:} \\
$\cdot$ if $U \subseteq \mathbb{R}^n$ is open and $C \subseteq \mathbb{R}^n$ is closed then $U \ C$ is open.  (Open \ Closed = Open)   \\
$\cdot$ if $U \subseteq \mathbb{R}^n$ is closed and $C \subseteq \mathbb{R}^n$ is open then $U \ C$ is closed.  (Closed \ Open = Closed )  \\
$\cdot$ if $U_1 , \cdots \, U_k \subseteq \mathbb{R}^n $ are open then $U_1 \cup \cdots \cup U_k$ and $U_1 \cap \cdots \cap U_k$ are open. \\
$\cdot$ if $C_1 , \cdots \, C_k \subseteq \mathbb{R}^n $ are open then $C_1 \cup \cdots \cup C_k$ and $C_1 \cap \cdots \cap C_k$ are closed\footnote{intersezione e unione di sottoinsiemi chiusi (rispettivamente aperti) danno un sottoinsieme ancora chiuso (rispettivamente aperto).}. \\ 






%immagini:
% \begin{figure}[!ht]
%   \centering
%   \subfloat[Misurazioni prese in rampa a 50khz]{
%     \includegraphics[width=150mm]{50khz.jpg}
%   }
% \end{figure}

\pagebreak



\section{Allegati}
\label{sec:allegati}
\subsection{Dimostrazione 1}
\label{sec:Dimostrazione 1}













\end{document} 