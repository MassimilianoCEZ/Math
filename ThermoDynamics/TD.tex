%This is Massimiliano's latex template 
\documentclass[a4paper,11pt]{article}

\usepackage{fancyhdr}



\usepackage[english,italian]{babel}


\textwidth16cm \textheight24cm \topmargin0mm \headheight10mm
\headsep5mm \oddsidemargin0mm \evensidemargin0mm
\parindent0mm
\usepackage{bettelini}
\usepackage{subfig}
\usepackage{amsmath}
\usepackage{parskip}
\usepackage{dsfont}
\usepackage{fullpage}
\usepackage{amssymb}
\usepackage{tikz,pgfplots}
\usepackage{cancel}
\usepackage{lmodern}
\usepackage{gensymb}
\usepackage[T1]{fontenc}
\usepackage{amsfonts}
\usepackage{theorem}
\usepackage{psfrag}
\usepackage{color}
\usepackage{graphicx}
\usepackage{hyperref}
\usepackage{wrapfig}  
%\usepackage[all]{hypcap}
\usepackage{keystroke}
\usepackage{etoolbox}
\usepackage{babel}
\usepackage{siunitx}
\usepackage{float}










\usepackage[AMRA, 30]{masterfrontpage}

\usepackage{blindtext}

\usetikzlibrary{3d}


\pgfplotsset{
  compat=1.13,
}

\makeatletter
\patchcmd{\tableofcontents}{\@starttoc{toc}}{\hypertarget{totoc}{}\@starttoc{toc}}{}{}
\makeatother

\usepackage{fancyvrb}




\usepackage[backend=bibtex]{biblatex}
\addbibresource{references.bib}

\hypersetup{
    colorlinks=true,
    linkcolor=black,
    urlcolor=blue,
    pdftitle={Lavoro per il Concorso nazionale di Scienza e gioventù},
    pdfpagemode=FullScreen,
}
\VerbatimFootnotes
\newcommand{\overbar}[1]{\mkern 1.5mu\overline{\mkern-1.5mu#1\mkern-1.5mu}\mkern 1.5mu}





% #1    x coordinate.
\newcommand{\vandbarrier}[1]{
    \node[scale = 0.5] at (#1, 0.5) {\text{(Barriera)}};%
}

% #1    x coordinate.
\newcommand{\vabove}[1]{%
    \node[anchor = south] at (#1, 2) {\scriptsize $V = \infty$};%
}




\usetikzlibrary{decorations.pathmorphing,decorations.markings,calc} % for random steps & snake
\usetikzlibrary{arrows.meta} % for arrow size
\tikzset{>=latex} % for LaTeX arrow head
\tikzstyle{radiation}=[-{Latex[length=2,width=1.5]},red!95!black!50,opacity=0.7,very thin,decorate,
                       decoration={snake,amplitude=0.7,segment length=2,post length=2}]






\tikzset{>=latex}





          \pgfdeclareverticalshading{rainbow}{100bp}{
              color(0bp)=(red); color(25bp)=(red); color(35bp)=(yellow);
              color(45bp)=(green); color(55bp)=(cyan); color(65bp)=(blue);
              color(75bp)=(violet); color(100bp)=(violet)
            }
            
            \pgfdeclareverticalshading{blackbody}{100}{
              rgb(0)=(0,0,0);
              rgb(25)=(0,0,0);
              rgb(25+50/11*1)=(1,0.0337,0);
              rgb(25+50/11*2)=(1,0.2647,0.0033);
              rgb(25+50/11*3)=(1,0.4870,0.1411);
              rgb(25+50/11*4)=(1,0.6636,0.3583);
              rgb(25+50/11*5)=(1,0.7992,0.6045);
              rgb(25+50/11*6)=(1,0.9019,0.8473);
              rgb(25+50/11*6.5)=(1,1,1);
              rgb(25+50/11*7)=(0.9337,0.9150,1);
              rgb(25+50/11*8)=(0.7874,0.8187,1);
              rgb(25+50/11*9)=(0.6925,0.7522,1);
              rgb(25+50/11*10)=(0.6268,0.7039,1);
              rgb(75)=(0.3277,0.5022,1);
              rgb(100)=(0.3277,0.5022,1)
            }


            


\include{settings_epfl.tex}



\let\oldAA\AA
\renewcommand{\AA}{\text{\normalfont\oldAA}}


%QUICK_VECTORS  https://tex.stackexchange.com/questions/200507/inline-row-vectors-with-smallmatrix
\newcommand{\icol}[1]{% inline column vector
  \left(\begin{smallmatrix}#1\end{smallmatrix}\right)%
}

\newcommand{\irow}[1]{% inline row vector
  \left(\begin{smallmatrix}#1\end{smallmatrix}\right)%
}


\renewcommand{\iff}{
  \xleftrightarrow{\text{iff}}
}

\newcommand{\bk}[1]{
  \langle {#1} \rangle
}

\renewcommand{\mod}[1]{
  \left\lVert {#1}\right\rVert_2
}

\newcommand{\abs}[1]{
  \left\lvert {#1} \right\rvert
}





\begin{document}


\title{Thermodynamics}
\subtitle{from chemistry to properties}
\author{Ferrulli Massimiliano}  
% SE VUOI LA COPERTINA: %
   % \masterfrontpage

\maketitle





\tableofcontents





\pagebreak



\section{Week1}
Z: Number of protons (protons and neutrons have roughly the same mass)\\
A: Atomic mass \\
\underline{Temperature:} Measures how hot / cold a sistem is, it's a state variable, it's relative and absolute temperature scales. \\
 \underline{Relative T scales:}\\
 Celsius: \\
 $\cdot$ 0  $\to$  freezing point of water \\
 $\cdot$ $ 100 \to $ boiling point of water \\
Farhenneit: \\
$\cdot$ $ 0 \to $ freezing point of brine \\
$\cdot$ $ 32 \to $ freezing point of water \\
$\cdot$ $ 98 \to $ Human body T \\
Conversion Celsius Farhenneit: 
$  \text{celsius} \cdot \frac{9}{5}  + 32 = \text{Farhenneit} $
\underline{ Absolute T scales}: \\
 Kelvin: \\
 $\cdot$ $ 0K $ is the lowest the T goes (always positive) $ = 273.15 C $  \\
 $\cdot$ Absolute zero: extrpolation of T of an ideal fas for zero volume \\
\underline{Therma equilibrium (TE)}: \\
Put two objects with different T in thermal contact $\to$ they will reach the same T said to be in thermal equilibrium. \\
$\cdot$ \underline{0th law of TD}: \\
if two systems are in TE with a third system, then they are in TE with each other. 


%immagini:
 \begin{figure}[!ht]
   \centering
   \subfloat[0th law of TD]{
     \includegraphics[width=50mm]{images/0thlaw.png}
   }
 \end{figure}

\pagebreak

\subsection{Thermal Expansions}
\underline{Linear expansion:} \\
$$ \Delta l = \alpha l_0 \Delta T $$
where $\alpha$ $\left[ \frac{1}{K} \right] $ is the linear thermal expansion coefficient dependent on the material. \\
\underline{Volume expansion:} \\
$$ \Delta V = \beta V_0 \Delta T $$
where $\beta$ $\left[ \frac{1}{K} \right] $ is the volume thermal expansion coefficient dependent on the material. \\
For small $\Delta T$ we can use the following formula: 
$$V(t) = l(t)^3 = (l_o + l(t))^3 = (l_0 + \alpha l_0 \Delta T )^3 = l_0^3(1+\alpha \Delta T)^3 $$
This formula is only usable for small $\Delta T$  \\
\begin{align}
  \begin{aligned}
    V(t) & =l_0^3\left(1+3 \alpha \Delta T+3 \alpha^2 \Delta T^2+\alpha^3 \Delta T^3\right) \, , \, \left(3 \alpha^2 \Delta T^2+\alpha^3 \Delta T^3 \, \, \text{are very small} \right)  \\
    & \approx l_0^3(1+3 \alpha \Delta T) \\
    & =V_0+3 \alpha \Delta T V_0
  \end{aligned}
\end{align}

\begin{align*}
  \begin{gathered}
    \Delta V=3 \alpha \Delta T V_0=\beta \Delta T V_0 \\
    3 \alpha \approx \beta
    \end{gathered}
\end{align*}



\underline{Precise formula for lenght expansion:}
For an infinitesimaly small extension $dl$:
\begin{align}
  \begin{aligned}
    & d l=\alpha l d T \\
    & \int \frac{d l}{l}=\int \alpha d T  \\
    & \log (l)=\alpha T+C \\
    & l=\exp (\alpha T) \cdot C
  \end{aligned}
\end{align}
with initial conditions: 
$$ l = e^{\alpha \Delta T} \cdot l_0 $$

\subsection{Thermal stress}
Expansion of bodies: push on each other. How much is a "pushed - on " object compressed? \\
Juong's modulus E:
$$ \Delta l = \frac{1}{E} P l_o $$ 
\underline{Example:} 



\pagebreak



\section{Allegati}
\label{sec:allegati}
\subsection{Dimostrazione 1}
\label{sec:Dimostrazione 1}













\end{document} 